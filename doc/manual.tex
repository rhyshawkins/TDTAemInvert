\documentclass[a4paper,12pt]{article}

\usepackage{natbib}
\usepackage{hyperref}

%\setlength{\textwidth}{425pt}
%\setlength{\oddsidemargin}{42pt}
%\setlength{\evensidemargin}{-15pt}

\setlength{\topmargin}{-15pt}

\begin{document}

\title{AEM Inversion User Manual}
\author{Rhys Hawkins}
\date{April 2018}

\maketitle

\tableofcontents

\section{Introduction}

This software uses a Bayesian Trans-dimensional Tree\citep{Hawkins:2015:A} approach with a
wavelet parameterisation for the inversion of 2D AEM Profile. This software
was used for the results published in \citet{Hawkins:2017:A}.


The software allows both hierarchical error estimation where there are
uncertainties in the error level on input observations and
hierarchical prior width. These two abilities combined allow the error
level and prior to be solved for as part of the inversion, resulting
in a more automated inversion.

Included is a parallel version that allows any combination of parallel
chains, parallel forward model evaluation and Parallel
Tempering\citep{Sambridge:2014:A}.

This software was developed by Rhys Hawkins as part of research leading
towards a PhD\citep{Hawkins:2018:A} at the Australian National University
and is copyright 2017 Rhys Hawkins and released under a GPL v3 Licence.

\section{Prerequisites}

It assumed that the software will be run on a Unix like system. This
software was developed and tested on Linux. The software requires a
recent GNU c++ compiler as some of the libraries use C++ templates
which are not compatible with older compilers.

The external libraries required by this software are

\begin{description}
\item[GSL] The GNU scientific library (2.4)
\item[GMP] The GNU arbitrary precision library (6.1)
\item[OpenMPI] Version 1.6 and 2.1 have been used.
\end{description}

\section{Installation}

The source code and supporting files are contained in the {\tt TDTAemInvert.tar.gz}
file, to extract and compile the code, use the following steps:

\begin{verbatim}
tar -xzf TDTAemInvert.tar.gz
cd TDTAemInvert
make
\end{verbatim}

And the executables will be in the current directory.

\section{Tutorial Introduction}

\subsection{Generating Synthetic Tests}

This section describes a simple walkthrough of running the code on
a synthetic data. 

\subsubsection{Create a synthetic image}

The starting point for this tutorial inversion is a true
model in the form of an image. This image needs to be a power
of 2 in width (laterally) and height (depth). For this example
here, we use the same synthetic model as in \citet{Hawkins:2017:A}.

This synthetic model is generated by applying a Gaussian filter
over a simple layered model. In the tutorial subdirectory, this
is achieved using a python script:

\begin{verbatim}
python2 smoothimage.py -i synthetic_pixel -o synthetic_smooth -s 1.5
\end{verbatim}

where the {\tt -s 1.5} is the parameter controlling the degree of
smoothing.

This step can be replaced with any other process to generate
your own synthetic test models, however it is assumed that
the image file is named ``synthetic\_smooth'' from herein.

\subsubsection{Converting and rescaling the synthetic image}

The raw image created in the previous step needs to be converted
into a different format for creation of synthetic AEM observations
and scaled to appropriate values for conductivity. This is
achieved by running the convertimage.py script as follows:

\begin{verbatim}
python2 convertimage.py -i synthetic_smooth \
  -o syntheticstudy.image \
  -r syntheticstudy.rawimage \
  -d 200.0 \
  --min 0.05 --max 0.20
\end{verbatim}

The {\tt -d 200.0} specifies the depth to halfspace of the
model, and the {\tt --min} and {\tt --max} specify the
range of conductivities that the model is rescaled to.

\subsubsection{Creating a synthetic flight path}

For the generation of AEM observations, the forward model
needs many parameters from the flight path of the sensor.
These can be created using the {\tt mksyntheticflightpath}
program which generates random walk flight paths.

\begin{verbatim}
../mksyntheticflightpath -o standard16x16.path \
        -N 16 \
        -e 30.0 -E 2.5 \
        -p -3.2 -P 2.0 \
        -r 0.0 -R 2.0 \
        -x -12.5 -X 0.1 \
        -z 2.0 -Z 0.1
\end{verbatim}

The {\tt -N 16} specifies the number of points in the flight path
which must match the width or lateral dimension of the synthetic
image. The other parameters represent the mean and standard deviation
of the Gaussian distribution of the parameters described below.

\begin{description}
\item[e/E] Height mean and standard deviation (m)
\item[p/P] Pitch mean and standard deviation (deg)
\item[r/R] Roll mean and standard deviation (deg)
\item[x/X] dx mean and standard deviation (m)
\item[z/Z] dz mean and standard deviation (m)
\end{description}

\subsubsection{Creating the synthetic observations}

Finally, with a true image of the subsurface and flight path, a set of
synthetic observations can be constructed with the {\tt mksyntheticobservations}
program. Additional files required (included in the distribution) are STM
files which describe the sensor used (see \citet{GAAEM:2016:A} for more details).

\begin{verbatim}
../mksyntheticobservations -i syntheticstudy.image \
  -I standard16x16.path \
  -S ../stm/SkytemLM-BHMAR.stm \
  -N ../noise_models/brodienoiseLM.txt
  -S ../stm/SkytemHM-BHMAR.stm \
  -N ../noise_models/brodienoiseHM.txt
  -o syntheticstudy.obs \
  -O syntheticstudy.true
\end{verbatim}

In this study, there are two moment sensors (low and high) and an STM
and noise model are required for each. These must be passed with {\tt -S}
for the STM files and {\tt -N} for the noise model and in the same order,
i.e Low STM then High STM and Low noise model the High noise model.

\subsubsection{Validation}


\begin{verbatim}

\end{verbatim}

\end{document}
